\documentclass{report}
\usepackage[english, ukrainian]{babel}
\usepackage{amsmath}
\usepackage{systeme}
\usepackage{graphicx}
\usepackage{titlesec}
\usepackage{listings}
\usepackage{xcolor}
\usepackage{hyperref}
\usepackage{verbatim} 
\usepackage{float}


\titleformat{\section}[block]{\Large\bfseries\filcenter}{}{1em}{}

\graphicspath{ {.} }

\setlength{\parindent}{4em}
\setlength{\parskip}{1em}
\renewcommand{\baselinestretch}{1.5}

\definecolor{codegreen}{rgb}{0,0.6,0}
\definecolor{codegray}{rgb}{0.5,0.5,0.5}
\definecolor{codepurple}{rgb}{0.58,0,0.82}
\definecolor{backcolour}{rgb}{0.95,0.95,0.92}

\hypersetup{
    colorlinks=true,
    linkcolor=blue,
    filecolor=magenta,      
    urlcolor=cyan,
    pdfpagemode=FullScreen,
    }
\urlstyle{same}

\lstdefinestyle{mystyle}{
    backgroundcolor=\color{backcolour},   
    commentstyle=\color{codegreen},
    keywordstyle=\color{magenta},
    numberstyle=\tiny\color{codegray},
    stringstyle=\color{codepurple},
    basicstyle=\ttfamily\footnotesize,
    breakatwhitespace=false,         
    breaklines=true,                 
    captionpos=b,                    
    keepspaces=true,                 
    numbers=left,                    
    numbersep=5pt,                  
    showspaces=false,                
    showstringspaces=false,
    showtabs=false,                  
    tabsize=2
}
\lstset{style=mystyle}

\begin{document}

\title{Чисельні методи\linebreakЛабораторна робота №3\\Варіант №3}
\author{Божко Роман Вячеславович, TP-02}
\date{\today}

\maketitle

\section*{Завдання за варіантом}
$f(x) = x^4 - 3x^3 + x^2 - 2x - 4 = 0$

\section*{Допрограмовий етап}
\subsection*{Теорема про верхню і нижню границі}
$f(x) = x^4 - 3x^3 + x^2 - 2x - 4 = 0 \to A = 4, k = 0$\newline
$R = 1 + \sqrt[0]{4} = 2 \mbox{ - верхня межа}$\newline
$f(-x) = x^4 + 3x^3 + x^2 + 2x - 4 = 0 \to A = 4, k = 0$\newline
$-R = 1 + \sqrt[0]{4} = 2, R = -2 \mbox{ - нижня межа}$\newline
Отримані межі: $x \in \left[-2, 2\right]$
\subsection*{Теорема Гюа}
$k = 3 \to 9 < 1 \cdot 1$\newline
$k = 2 \to 1 < -2 \cdot (-3), 1 < 6 \to \mbox{комплексні корені існують}$\newline
$k = 1 \to -2 < -4 \cdot 1$
\subsection*{Теорема про границі усіх коренів рівняння}
$A = |-4| = 4, B = |-4| = 4$\newline
$\frac{4}{4 + 4} \leq |x^{*}| \leq \frac{1 + 4}{1}$\newline
$\frac{1}{2} \leq |x^{*}| \leq 5 \to x \in \left[0.5, 5\right]$
\subsection*{Теорема Штурма про чередування коренів}
$f_0 = f(x) = x^4 - 3x^3 + x^2 - 2x - 4 = 0$\newline
$f_1 = f'(x) = 4x^3 - 9x^2 + 2x - 2 = 0$\newline
$f_2 = -(f_0 \bmod f_1) = 1.1875 \cdot x^2 + 1.125 \cdot x + 4.375$\newline
$f_3 = -(f_1 \bmod f_2) = 0.6204 \cdot x - 45.1191$\newline
$f_4 = -(f_2 \bmod f_3) = -6364.9273$

Остаточні межі: $x \in \left[-2, 5\right]$
\begin{center}
\begin{tabular}{||c c c||} 
 \hline
 № & a = -2 & b = 5 \\ [0.5ex] 
 \hline\hline
 $f_0$ & + & +\\ 
 \hline
 $f_1$ & - & +\\
 \hline
 $f_2$ & + & +\\
 \hline
 $f_3$ & - & -\\
 \hline
 $f_4$ & - & -\\ [1ex] 
 \hline
\end{tabular}
\end{center}
$N_a - N_b = 3 - 1 = 2 \to $ кількість дійсних коренів рівняння.

\includegraphics[width=\textwidth,keepaspectratio]{sturm_table.png}

За таблицею знайдемо інтервали на яких відбувається зміна знаків. Всього їх має бути 2. Це інтервали: $\left[-1, 0\right] , \left[3, 4\right]$. На них ми і знайдемо корені рівняння.

\section*{Метод бісекції}
\lstinputlisting[language=Python, caption=\mbox{Програма для уточнення коренів за методом бісекції}]{./nonlinmeth/bisection.py}

\section*{Метод хорд}
\lstinputlisting[language=Python, caption=\mbox{Програма для уточнення коренів за методом хорд}]{./nonlinmeth/secant.py}

\section*{Метод дотичних}
\lstinputlisting[language=Python, caption=\mbox{Програма для уточнення коренів за методом дотичних}]{./nonlinmeth/newton.py}

\section*{Лістинг програми}
\lstinputlisting[language=Python, caption=\mbox{Програма, що містить виклики методів уточнення коренів}]{./main.py}

\section*{Результат роботи програми з уточнення коренів нелінійного алгебраїчного рівняння}\
\includegraphics[width=\textwidth,keepaspectratio]{results.png}

\section*{Висновки}
Метод бісекції може бути застосований, якщо функція неперервна на визначеному відрізку. При використанні методів хорд і дотичних необхідно аби перша і друга похідні були сталі за знаком на визначеному відрізку. Таким чином, метод бісекції є найбільш універсальним, проте він сходиться найповільніше серед трьох розглянутих. На противагу цьому метод дотичних має найбільшу швидкість сходимості - квадратичну. Метод хорд сходиться повільніше, але для нього не потрібно знаходити значення похідної.

\end{document}