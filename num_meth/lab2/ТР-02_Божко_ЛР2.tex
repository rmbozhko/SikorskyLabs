\documentclass{report}
\usepackage[english, ukrainian]{babel}
\usepackage{amsmath}
\usepackage{systeme}
\usepackage{graphicx}
\usepackage{titlesec}
\usepackage{listings}
\usepackage{xcolor}
\usepackage{hyperref}
\usepackage{verbatim} 
\usepackage{float}


\titleformat{\section}[block]{\Large\bfseries\filcenter}{}{1em}{}

\graphicspath{ {.} }

\setlength{\parindent}{4em}
\setlength{\parskip}{1em}
\renewcommand{\baselinestretch}{1.5}

\definecolor{codegreen}{rgb}{0,0.6,0}
\definecolor{codegray}{rgb}{0.5,0.5,0.5}
\definecolor{codepurple}{rgb}{0.58,0,0.82}
\definecolor{backcolour}{rgb}{0.95,0.95,0.92}

\hypersetup{
    colorlinks=true,
    linkcolor=blue,
    filecolor=magenta,      
    urlcolor=cyan,
    pdfpagemode=FullScreen,
    }
\urlstyle{same}

\lstdefinestyle{mystyle}{
    backgroundcolor=\color{backcolour},   
    commentstyle=\color{codegreen},
    keywordstyle=\color{magenta},
    numberstyle=\tiny\color{codegray},
    stringstyle=\color{codepurple},
    basicstyle=\ttfamily\footnotesize,
    breakatwhitespace=false,         
    breaklines=true,                 
    captionpos=b,                    
    keepspaces=true,                 
    numbers=left,                    
    numbersep=5pt,                  
    showspaces=false,                
    showstringspaces=false,
    showtabs=false,                  
    tabsize=2
}
\lstset{style=mystyle}

\begin{document}

\title{Чисельні методи\linebreakЛабораторна робота №2\\Варіант №3}
\author{Божко Роман Вячеславович, TP-02}
\date{\today}

\maketitle

\section*{Завдання}
$A = 
\begin{pmatrix}
5,68 & 1,12 & 0,95 & 1,32 & 0,83\\
1,12 & 3,78 & 2,12 & 0,57 & 0,91\\
0,95 & 2,12 & 6,63 & 1,29 & 1,57\\
1,32 & 0,57 & 1,29 & 4,07 & 1,25\\
0,83 & 0,91 & 1,57 & 1,25 & 5,71\\
\end{pmatrix}, b = 
\begin{pmatrix}
6,89\\
3,21\\
3,58\\
6,25\\
5,65\\
\end{pmatrix}
$\par
Якщо матриця не є матрицею із діагональною перевагою, привести систему до
еквівалентної, у якій є діагональна перевага (письмово). Можна, наприклад, провести одну ітерацію метода Гауса, зкомбінувавши рядки з метою отримати нульовий недіагональний елемент у стовпчику. Розробити програму, що реалізує розв’язання за ітераційним методом, який відповідає заданому варіанту. Обчислення проводити з з кількістю значущих цифр $m = 6$. Для кожної ітерації розраховувати нев’язку $r = b - Ax$, де $x$ - отриманий розв’язок.\par
	Розв’язати задану систему рівнянь за допомогою програмного забезпечення Mathcad. Навести результат перевірки: вектор нев’язки $r = b - Ax_m$, де $x_m$ - отриманий у Mathcad розв’язок.\par
	Порівняти корені рівнянь, отримані у Mathcad, із власними результатами за допомогою методу середньоквадратичної похибки:
\[\delta=\sqrt{\frac{1}{n}\sum_{k=1}^{n} (x_k - x_{mk})^2}\],
де $x$ - отриманий у програмі розв’язок, $x_m$ - отриманий у Mathcad розв’язок.

\section*{Теоретичні відомості}
	Системи лінійних алгебраїчних рівнянь можна розв‘язувати за допомогою як прямих, так і ітераційних методів.\par
	Метод розв‘язання СЛАУ відносять до класу прямих, або точних, якщо за умови відсутності округлень він дає точний розв‘язок задачі після скінченного числа арифметичних і логічних операцій. До цих методів належать метод Гаусса і його модифікації, метод відбиття (Хаусхолдера), метод обертань (Гівенса), метод квадратного кореня і метод Холесського.\par
	Ітераційні методи розв‘язання СЛАУ - це методи наближеного \linebreak розв'язування, що базуються на послідовному наближенні до розв‘язку шляхом багатократного застосування деякої обчислювальної процедури, при цьому вихідними даними для кожної наступної процедури є результати застосування попередніх процедур. Наслідком такого ітераційного процесу є послідовність, яка при виконанні деяких умов збігається до розв‘язку задачі.\par
	Для систем середньої вимірності часто більш привабливими є прямі методи. Ітераційні методи застосовуються головним чином для розв‘язування складних задач великої вимірності, для яких внаслідок обмежень, що накладаються на об‘єм робочої пам‘яті і число арифметичних операцій, використання прямих методів виявляється достатньо важким або значно менше ефективним. Наприклад, ітераційні методи, як правило, застосовуються для розв‘язання задач з трьома просторовими змінними, задач, що включають системи нелінійних рівнянь, задач, що виникають при дискретизації систем рівнянь в частинних похідних, а також для розв‘язування нестаціонарних задач з більш ніж однією просторовою змінною.\par
\section*{Приведення до матриці з діагональною перевагою}
Оскільки початкова матриця не є матрицею з діагональною перевагою, необхідно привести систему до еквівалетної, котра матиме діагональну перевагу.\par
$ \bar{A} = \begin{pmatrix}
	5,68 & 1,12 & 0,95 & 1,32 & 0,83 & 6,89\\
	1,12 & 3,78 & 2,12 & 0,57 & 0,91 & 3,21\\
	0,95 & 2,12 & 6,63 & 1,29 & 1,57 & 3,58\\
	1,32 & 0,57 & 1,29 & 4,07 & 1,25 & 6,25\\
	0,83 & 0,91 & 1,57 & 1,25 & 5,71 & 5,65
	\end{pmatrix} \sim \\
	\begin{pmatrix}
	5,68 & 1,12 & 0,95 & 1,32 & 0,83 & 6,89\\
	0 & -20,21600 & -10,9776 & -1,7592 & -4,2392 & -10,516\\
	0 & -10,9776 & -36,7559 & -6,0732 & -8,1291 & -13,788\\
	0 & -1,7592 & -6,0732 & -21,3752 & -6,0044 & -26,405\\
	0 & -4,2392 & -8,1291 & -6,0044 & -31,7439 & -26,373
	\end{pmatrix}\\
$\par
При переході до еквівалетної системи були виконані наступні дії:
\begin{itemize}
  \item Від 1-го рядка домноженого на 1,12 було віднято 2-й рядок домножений на 5,68
  \item Від 1-го рядка домноженого на 0,95 було віднято 3-й рядок домножений на 5,68
  \item Від 1-го рядка домноженого на 1,32 було віднято 4-й рядок домножений на 5,68
  \item Від 1-го рядка домноженого на 0,83 було віднято 5-й рядок домножений на 5,68
\end{itemize}

\section*{Лістинг програми}
\lstinputlisting[language=Python, caption=\mbox{Код, що реалізує метод Зейделя}]{./seidel.py}

\section*{Вивід коренів системи та векторів нев'язок на етапі кожної ітерації в консоль}\
\includegraphics[width=\textwidth,keepaspectratio]{results.png}
\includegraphics[width=\textwidth,keepaspectratio]{residual.png}


\section*{Вектор нев’язки}
$r = b - Ax = 
\begin{pmatrix}
6,89\\
-10,516\\
-13,788\\
-26,405\\
-26,373
\end{pmatrix} - \begin{pmatrix}
5,68 & 1,12 & 0,95 & 1,32 & 0,83\\
0 & -20,21600 & -10,9776 & -1,7592 & -4,2392\\
0 & -10,9776 & -36,7559 & -6,0732 & -8,1291\\
0 & -1,7592 & -6,0732 & -21,3752 & -6,0044\\
0 & -4,2392 & -8,1291 & -6,0044 & -31,7439
\end{pmatrix}\begin{pmatrix}
0,82374\\
0,317346\\
-0,024843\\
1,048713\\
0,596422
\end{pmatrix} = \begin{pmatrix}
-4,904359e^{-8}\\
-1,289813e^{-7}\\
-3,758611e^{-9}\\
-2,998069e^{-8}\\
0
\end{pmatrix}$


\section*{Копія розв’язку задачі у Numpy\\Вектор нев’язки для цього розв’язку}
\lstinputlisting[language=Python, caption=\mbox{Програма-розв'язк, що використовує пакет Numpy}]{./lab2.py}

Розв'язок системи за допомогою Numpy: $x_m = \begin{pmatrix}0,8237398\\0,31734588\\-0,02484263\\1,04871263\\0,59642216\end{pmatrix}\\
r = b - Ax_m =
\begin{pmatrix}
6,89\\
-10,516\\
-13,788\\
-26,405\\
-26,373
\end{pmatrix} - \begin{pmatrix}
5,68 & 1,12 & 0,95 & 1,32 & 0,83\\
0 & -20,21600 & -10,9776 & -1,7592 & -4,2392\\
0 & -10,9776 & -36,7559 & -6,0732 & -8,1291\\
0 & -1,7592 & -6,0732 & -21,3752 & -6,0044\\
0 & -4,2392 & -8,1291 & -6,0044 & -31,7439
\end{pmatrix}\begin{pmatrix}0,8237398\\0,31734588\\-0,02484263\\1,04871263\\0,59642216\end{pmatrix} = \begin{pmatrix}
0\\ 0\\ 0\\ 0\\0\\
\end{pmatrix}$

\section*{Порівняння власного розв’язку та розв’язку, отриманого у Numpy}
\[\delta=\sqrt{\frac{1}{n}\sum_{k=1}^{n} (x_k - x_{mk})^2} = 3.345384e^{-17}\] %3.345384099418549e^{-17}
Розв'язок збігається в усіх коренях, проте значення отримані при розрахунку в Numpy є більш точними, бо для представлення чисел з плаваючою точкою використовуються більш вмісткі типи даних. У зв'язку з цим середньоквадратичне відхилення не дорівнює нулеві.


\end{document}