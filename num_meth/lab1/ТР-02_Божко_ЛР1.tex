\documentclass{report}
\usepackage[english, ukrainian]{babel}
\usepackage{amsmath}
\usepackage{systeme}
\usepackage{graphicx}
\usepackage{titlesec}
\usepackage{listings}
\usepackage{xcolor}
\usepackage{hyperref}
\usepackage{verbatim} 
\usepackage{float}


\titleformat{\section}[block]{\Large\bfseries\filcenter}{}{1em}{}

\graphicspath{ {.} }

\setlength{\parindent}{4em}
\setlength{\parskip}{1em}
\renewcommand{\baselinestretch}{1.5}

\definecolor{codegreen}{rgb}{0,0.6,0}
\definecolor{codegray}{rgb}{0.5,0.5,0.5}
\definecolor{codepurple}{rgb}{0.58,0,0.82}
\definecolor{backcolour}{rgb}{0.95,0.95,0.92}

\hypersetup{
    colorlinks=true,
    linkcolor=blue,
    filecolor=magenta,      
    urlcolor=cyan,
    pdfpagemode=FullScreen,
    }
\urlstyle{same}

\lstdefinestyle{mystyle}{
    backgroundcolor=\color{backcolour},   
    commentstyle=\color{codegreen},
    keywordstyle=\color{magenta},
    numberstyle=\tiny\color{codegray},
    stringstyle=\color{codepurple},
    basicstyle=\ttfamily\footnotesize,
    breakatwhitespace=false,         
    breaklines=true,                 
    captionpos=b,                    
    keepspaces=true,                 
    numbers=left,                    
    numbersep=5pt,                  
    showspaces=false,                
    showstringspaces=false,
    showtabs=false,                  
    tabsize=2
}
\lstset{style=mystyle}

\begin{document}

\title{Чисельні методи\linebreakЛабораторна робота №1\\Варіант №3}
\author{Божко Роман Вячеславович, TP-02}
\date{\today}

\maketitle

\section*{Завдання}
$A = 
\begin{pmatrix}
3,81 && 0,25 && 1,28 && 1,75\\
2,25 && 1,32 && 5,58 && 0,49\\
5,31 && 7,28 && 0,98 && 1,04\\
10,39 && 2,45 && 3,35 && 2,28\\
\end{pmatrix}, b = 
\begin{pmatrix}
4,21\\
7,47\\
2,38\\
11,48\\
\end{pmatrix}
$\par
Розв’язати систему рівнянь з кількістю значущих цифр $m = 6$. Якщо матриця системи симетрична, то розв’язання проводити за методом квадратних коренів, якщо матриця системи несиметрична, то використати метод Гауса. Вивести всі проміжні результати (матриці А, що отримані в ході прямого ходу методу Гауса, матрицю зворотного ходу методу Гауса, або матрицю Т та вектор y для методу квадратних коренів), та розв’язок системи. Навести результат перевірки: вектор нев’язки $r = b - Ax$, де $x$ - отриманий розв’язок.\par
	Розв’язати задану систему рівнянь за допомогою програмного забезпечення Mathcad. Навести результат перевірки: вектор нев’язки $r = b - Ax_m$, де $x_m$ - отриманий у Mathcad розв’язок.\par
	Порівняти корені рівнянь, отримані у Mathcad, із власними результатами за допомогою методу середньоквадратичної похибки:
\[\delta=\sqrt{\frac{1}{n}\sum_{k=1}^{n} (x_k - x_{mk})^2}\],
де $x$ - отриманий у програмі розв’язок, $x_m$ - отриманий у Mathcad розв’язок.

\section*{Теоретичні відомості}
Прямий хід: Шляхом елементарних перетворень рядків (додавань до рядка іншого рядка, помноженого на число, і переставлянь рядків) матрицю приводять до верхньотрикутного вигляду (сходинчастого вигляду). З цього моменту починається зворотний хід. З останнього ненульового рівняння виражають кожну з базисних змінних через небазисні й підставляють до попередніх рівнянь. Повторюючи цю процедуру для всіх базисних змінних, отримують фундаментальний розв'язок.\par
Наприклад, треба виконати наступну послідовність перетворень (де на кожному кроці виконано декілька елементарних перетворень матриці), щоб фінальна утворена матриця прийняла свою унікальну скорочену рядкову ступінчасту форму.\par
У кожному рядку матриці, якщо рядок не складається із самих нулів, не нульове входження, яке знаходиться лівіше від усіх називають провідним коефіцієнтом цього рядка. Тому якщо два провідних коефіцієнти знаходяться в одному стовпці, тоді можна застосувати операцію над рядком типу 3 (див. вище) аби один з цих коефіцієнтів став нульовим. Далі, використавши операцію заміни рядків, завжди можна впорядкувати рядки таким чином, що для кожного не нульового рядка, провідний коефіцієнт знаходитиметься праворуч від провідного коефіцієнта рядка, що знаходиться вище. Якщо це так для всіх рядків, то говорять що матриця знаходиться у рядковій ступінчастій формі. Таким чином ліва нижня частина матриці містить лише нулі, і всі нульові рядки знаходяться нижче не нульових рядків. Слово «ступінчаста» використовується тут тому, оскільки можна вважати, що рядки матриці впорядковані за їх розміром, так що найбільший рядок знаходиться зверху, а найменший рядок — знизу.\par
\textbf{Алгоритм прямого ходу}:\par
Переберімо стовпчики матриці й здійснимо рядкові операції, щоб у кожному стовпчику:
\begin{itemize}
	\item діагональний елемент став дорівнювати одиниці;
	\item елементи під діагональним стали дорівнювати нулеві.
\end{itemize}
\textbf{Алгоритм зворотнього ходу}:\par
Переберімо стовпчики матриці у зворотному порядку й здійснімо рядкові операції, щоб у кожному стовпчику елементи над діагональним стали дорівнювати нулеві.



\section*{Лістинг програми}
\lstinputlisting[language=Java, caption=\mbox{Методи, що реалізують прямий і обернений ходи}]{./gauss/src/lab1/GaussianElimination.java}
Повна версія програми знаходиться за \href{https://github.com/rmbozhko/SikorskyLabs/tree/master/num_meth/lab1/gauss/src}{цим посиланням}

\section*{Проміжні результати та кінцевий результат}
\begin{figure}[H]
    \centering
    \includegraphics{results}
    \caption{Вивід результатів роботи програми в консоль}
    \label{pic:results}
\end{figure}
\begin{comment}
Роширена верхня трикутна матриця, що отримана в ході прямого ходу методу Гауса:
\[A =\begin{pmatrix}
3,81 && 0,25 && 1,28 && 1,75 && 4,21\\
0,00 && -1,98 && -8,16 && 0,92 && -8,43\\
0,00 && 0,00 && -8,39 && 0,51 && -9,43\\
0,00 && 0,00 && 0,00 && 2,41 && -0,92\\
\end{pmatrix}\]
Розв'язок системи: $x = \begin{pmatrix}
0,941074\\
-0,452852\\
1,100005\\
-0,383019\\
\end{pmatrix}$
\end{comment}

\section*{Вектор нев’язки}
$r = b - Ax = 
\begin{pmatrix}
4,21\\
7,47\\
2,38\\
11,48\\
\end{pmatrix} - \begin{pmatrix}
3,81 && 0,25 && 1,28 && 1,75\\
2,25 && 1,32 && 5,58 && 0,49\\
5,31 && 7,28 && 0,98 && 1,04\\
10,39 && 2,45 && 3,35 && 2,28\\
\end{pmatrix}\begin{pmatrix}
0,941074\\
-0,452852\\
1,100005\\
-0,383019\\
\end{pmatrix} = \begin{pmatrix}
0\\
-8.8817842e^{-16}\\
-8.8817842e^{-16}\\
0\\
\end{pmatrix}$


\section*{Копія розв’язку задачі у Numpy\\Вектор нев’язки для цього розв’язку}
\begin{lstlisting}[language=Python]
import numpy as np
    
A = np.array([[3.81, 0.25, 1.28, 1.75], [2.25, 1.32, 5.58, 0.49], [5.31, 7.28, 0.98, 1.04], [10.39, 2.45, 3.35, 2.28]])
B = np.array([4.21, 7.47, 2.38, 11.48])
X = np.linalg.solve(A,B)
\end{lstlisting}
Розв'язок системи за допомогою Numpy: $x_m = \begin{pmatrix}
0,94107373\\	
-0,45285249\\
1,1000052\\
-0,38301968\\
\end{pmatrix}\\
r = b - Ax_m = 
\begin{pmatrix}
4,21\\
7,47\\
2,38\\
11,48\\
\end{pmatrix} - \begin{pmatrix}
3,81 && 0,25 && 1,28 && 1,75\\
2,25 && 1,32 && 5,58 && 0,49\\
5,31 && 7,28 && 0,98 && 1,04\\
10,39 && 2,45 && 3,35 && 2,28\\
\end{pmatrix}\begin{pmatrix}
0,94107373\\
-0,45285249\\
1,1000052\\
-0,38301968\\
\end{pmatrix} = \begin{pmatrix}
0\\
-8.8817842e^{-16}\\
-8.8817842e^{-16}\\
0\\
\end{pmatrix}$

\section*{Порівняння власного розв’язку та розв’язку, отриманого у Numpy}
\[\delta=\sqrt{\frac{1}{n}\sum_{k=1}^{n} (x_k - x_{mk})^2} = 2.2498821869062463e^{-7}\]
Розв'язок збігається в усіх коренях, проте значення отримані при розрахунку в Numpy є більш точними, бо для представлення чисел з плаваючою точкою використовуються більш вмісткі типи даних. У зв'язку з цим середньоквадратичне відхилення не дорівнює нулеві.


\end{document}